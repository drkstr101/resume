%%%%%%%%%%%%%%%%%
% This is an sample CV template created using altacv.cls
% (v1.3, 10 May 2020) written by LianTze Lim (liantze@gmail.com). Now compiles with pdfLaTeX, XeLaTeX and LuaLaTeX.
%
%% It may be distributed and/or modified under the
%% conditions of the LaTeX Project Public License, either version 1.3
%% of this license or (at your option) any later version.
%% The latest version of this license is in
%%    http://www.latex-project.org/lppl.txt
%% and version 1.3 or later is part of all distributions of LaTeX
%% version 2003/12/01 or later.
%%%%%%%%%%%%%%%%

%% If you need to pass whatever options to xcolor
\PassOptionsToPackage{dvipsnames}{xcolor}

%% If you are using \orcid or academicons
%% icons, make sure you have the academicons
%% option here, and compile with XeLaTeX
%% or LuaLaTeX.
% \documentclass[10pt,a4paper,academicons]{altacv}

%% Use the "normalphoto" option if you want a normal photo instead of cropped to a circle
% \documentclass[10pt,a4paper,normalphoto]{altacv}

\documentclass[10pt,a4paper,ragged2e,withhyper]{altacv}

%% AltaCV uses the fontawesome5 and academicons fonts
%% and packages.
%% See http://texdoc.net/pkg/fontawesome5 and http://texdoc.net/pkg/academicons for full list of symbols. You MUST compile with XeLaTeX or LuaLaTeX if you want to use academicons.

\usepackage{preamble}

%% main.bib contains your publications
\addbibresource{main.bib}

\begin{document}
\name{Aaron R Miller}
\tagline{Enterprise Software Developer}
%% You can add multiple photos on the left or right
\photoR{2.8cm}{drkstr101}
% \photoL{2.5cm}{Yacht_High,Suitcase_High}

\personalinfo{
  % Not all of these are required!
  \email{aaron.miller@waweb.io}
  \phone{509-491-3153}
  \mailaddress{101, S Kellogg st, 99336, USA}
  \location{Kennewick, WA}
  \homepage{www.waweb.io}
  % \twitter{@twitterhandle}
  \github{drkstr101}
  \NewInfoField{gitlab}{\faGitlab}[https://gitlab.com/]
  \gitlab{aaron.miller}
  \linkedin{aaron-miller-452244173}
  %% You MUST add the academicons option to \documentclass, then compile with LuaLaTeX or XeLaTeX, if you want to use \orcid or other academicons commands.
  % \orcid{0000-0000-0000-0000}
  %% You can add your own arbtrary detail with
  %% \printinfo{symbol}{detail}[optional hyperlink prefix]
  % \printinfo{\faPaw}{Hey ho!}[https://example.com/]
  %% Or you can declare your own field with
  %% \NewInfoFiled{fieldname}{symbol}[optional hyperlink prefix] and use it:
}

\makecvheader
%% Depending on your tastes, you may want to make fonts of itemize environments slightly smaller
% \AtBeginEnvironment{itemize}{\small}

%% Set the left/right column width ratio to 6:4.
\columnratio{0.6}

% Start a 2-column paracol. Both the left and right columns will automatically
% break across pages if things get too long.
\begin{paracol}{2}

\cvsection{Clifton Strengths™}

% Adapted from @Jake's answer from http://tex.stackexchange.com/a/82729/226
% \wheelchart{outer radius}{inner radius}{
% comma-separated list of value/text width/color/detail}
\wheelchart{1.5cm}{0.5cm}{%
  6/8em/accent!30/{Ideation},
  3/8em/accent!40/Learner,
  8/8em/accent!60/Relator,
  2/10em/accent!10/Connectedness,
  5/6em/accent!20/Futuristic
}

\bigskip

\cvsection{Recent Projects}

\cvevent{Xtext}{Contributer}{May 2020 -- Ongoing}{}
SDK for developing programming and domain-specific language editors

\divider

\cvevent{Natural}{Maintainer}{April 2020 -- Ongoing}{}
Eclipse plugin editor for Cucumber and JBehave languages

\divider

\cvevent{Earth New Media}{Owner}{October 2019 -- Ongoing}{}
Federated social networking platform (Elaboration Phase) 

\divider

\cvevent{CartoonifyIt}{Owner}{September 2019 -- Ongoing}{}
Demo Augmented Reality (AR) application built with Rust and the Android NDK (Construction Phase) 

\divider

\cvevent{WaWeb Site Plugin}{Owner}{July 2019 -- Ongoing}{}
Integrated gradle plugins for static site development and publication

% use ONLY \newpage if you want to force a page break for
% ONLY the current column
\newpage

\cvsection{Accomplishments}

% \cvsection{Projects}
% 
\cvevent{Project 1}{Funding agency/institution}{}{}
\begin{itemize}
\item Details
\end{itemize}

\divider

\cvevent{Project 2}{Funding agency/institution}{April 2020 -- Ongoing}{}
A short abstract would also work.


\bigskip

\cvsection{Work History}

\cvevent{Eclipse Xtext}{Company 1}{April 2020 -- Ongoing}{Remote}
\begin{itemize}
\item Job description 1
\item Job description 2
\end{itemize}

\divider

\cvevent{Natural}{Company 2}{Month 20XX -- Ongoing}{Location}
\begin{itemize}
\item Job description 1
\item Job description 2
\end{itemize}

\nocite{*}

% \printbibliography[heading=pubtype,title={\printinfo{\faBook}{Books}},type=book]
% 
% \divider
% 
% \printbibliography[heading=pubtype,title={\printinfo{\faFile*[regular]}{Journal Articles}},type=article]
% 
% \divider
% 
% \printbibliography[heading=pubtype,title={\printinfo{\faUsers}{Conference Proceedings}},type=inproceedings]

%% Switch to the right column. This will now automatically move to the second
%% page if the content is too long.
\switchcolumn


\begin{quote}
``Learn from yesterday, live for today, hope for tomorrow. The important thing is not to stop questioning.''
— Albert Einstein
\end{quote}

\medskip

% \cvsection{Most Proud of}
% 
% \cvachievement{\faTrophy}{Fantastic Achievement}{and some details about it}
% 
% \divider
% 
% \cvachievement{\faHeartbeat}{Another achievement}{more details about it of course}
% 
% \divider
% 
% \cvachievement{\faHeartbeat}{Another achievement}{more details about it of course}

\cvsection{Top Skills}

\cvtag{Programming}
\cvtag{Leadership}
\cvtag{Systems}
\cvtag{Communication}
\cvtag{DevOps}
\cvtag{UI/UX Design}

\begin{minipage}{7.1cm}
	\includegraphics[width=\linewidth]{skillcloud.png}
\end{minipage}%

\medskip

% \divider\smallskip

% 
% \cvtag{C++}
% \cvtag{Embedded Systems}\\
% \cvtag{Statistical Analysis}

\cvsection{Languages}

\cvskill{Java/JDK}{5}
\divider

\cvskill{Groovy}{5}
\divider

\cvskill{JavaScript}{5}
\divider

\cvskill{Typescript}{4}
\divider

\cvskill{Python}{4}
\divider

\cvskill{C/C++}{3}
\divider

\cvskill{C\#/.NET}{3}
\divider

\cvskill{Kotlin}{2}
\divider

\cvskill{PHP}{2}
\divider

\cvskill{Rust}{1}

% Yeah I didn't spend too much time making all the
% spacing consistent... sorry. Use \smallskip, \medskip,
% \bigskip, \vpsace etc to make ajustments.
% \medskip
\newpage

% \divider

\cvsection{Referees}

% \cvref{name}{email}{mailing address}
\cvref{Prof.\ Alpha Beta}{Institute}{a.beta@university.edu}
{Address Line 1\\Address line 2}

\divider

\cvref{Prof.\ Gamma Delta}{Institute}{g.delta@university.edu}
{Address Line 1\\Address line 2}


\end{paracol}


\end{document}
